\documentclass[a4paper,11pt]{article}
% \linespread{2.}
\usepackage{latexsym}
\usepackage{amssymb}
\usepackage{amsbsy}
\usepackage{amsmath}
\usepackage[varg]{txfonts}
\usepackage{mathrsfs}
\usepackage{upgreek}
%\usepackage [latin1]{inputenc}
\usepackage{verbatim}
\usepackage{array}
\usepackage{color}
\pagestyle{plain}
\usepackage{graphicx}
\usepackage{hyperref}
\newcommand{\braket}[1]{\langle {#1} \rangle }
\newcommand{\ket}[1]{|{#1} \rangle }
\newcommand{\bra}[1]{\langle {#1}|}
\newcommand\idop{\mathds 1}
\DeclareMathAlphabet{\mathpzc}{OT1}{pzc}{m}{it}
\title{Optical Potential in a two--channels model}
\begin{document}
\maketitle
%\tableofcontents
\section{Schematic 2--channels model}
Let us  consider the example of the scattering between a proton and a nucleus $A$. The whole system can be described with relative $p-A$ coordinate $\mathbf{r}$ and the set of internal coordinates $\xi$ corresponding to the nucleons of $A$. The total hamiltonian is
\begin{equation}\label{eq1}
H(\xi,\mathbf{r})=H_{A}(\xi)+T(\mathbf{r})+V(\xi,\mathbf{r}).
\end{equation}
We are interested in describing  the elastic scattering between $p$ and $A$, corresponding to the wavefunction $\Psi_0(\xi,\mathbf{r})=\chi_0(\mathbf{r})\phi_0(\xi)$.
We will now try to see what happens when the elastic channels couples to just one more channel. Within this approximation, the wavefunction can be written as
\begin{equation}\label{eq2}
\Psi(\xi,\mathbf{r})=\chi_0(\mathbf{r})\phi_0(\xi)+\chi_1(\mathbf{r})\phi_1(\xi),
\end{equation}
with
\begin{eqnarray}\label{eq3}
\nonumber H_A(\xi)\phi_0(\xi)&=\varepsilon_0\phi_0(\xi)\\
H_A(\xi)\phi_1(\xi)&=\varepsilon_1\phi_1(\xi).
\end{eqnarray}
Then, the Schr\"{o}dinger equation
\begin{equation}\label{eq4}
(H-E)\Psi=0
\end{equation}
can be projected on the two internal wavefunctions $\phi_0(\mathbf{r}),\phi_1(\mathbf{r})$ and decomposed in the two coupled equations for the relative motion wavefunctions $\chi_0(\mathbf{r}),\chi_1(\mathbf{r})$
\begin{eqnarray}\label{eq5}
\nonumber \label{eq12}\left(T(\mathbf{r})+V_{00}(\mathbf{r})-\frac{\hbar^2k_0^2}{2 \mu}\right)\chi_0(\mathbf{r})&=-V_{01}(\mathbf{r})\chi_1(\mathbf{r})\\
\label{eq7} \left(T(\mathbf{r})+V_{11}(\mathbf{r})-\frac{\hbar^2k_1^2}{2 \mu}\right)\chi_1(\mathbf{r})&=-V_{10}(\mathbf{r})\chi_0(\mathbf{r}),
\end{eqnarray}
where $\tfrac{\hbar^2k_0^2}{2 \mu}=E-\varepsilon_0$, $\tfrac{\hbar^2k_1^2}{2 \mu}=E-\varepsilon_1$ and
\begin{equation}\label{eq6}
V_{ij}(\mathbf{r})=\int d\xi \phi^*_i(\xi)V(\xi,\mathbf{r})\phi_j(\xi).
\end{equation}
According to (\ref{eq7}), we can express $\chi_1(\mathbf{r})$ in the form

\begin{equation}\label{eq8}
\chi_1(\mathbf{r})=-\int d\mathbf{r}'G(\mathbf{r},\mathbf{r}')V_{10}(\mathbf{r}')\chi_0(\mathbf{r}'),
\end{equation}
where $G(\mathbf{r},\mathbf{r}')$ is the Green function obeying
\begin{equation}\label{eq9}
 \left(T(\mathbf{r})+V_{11}(\mathbf{r})-\frac{\hbar^2k_1^2}{2 \mu}\right)G(\mathbf{r},\mathbf{r}')=\delta(\mathbf{r}-\mathbf{r}').
\end{equation}
The Green function can be written as

\begin{equation}\label{eq10}
G(\mathbf{r},\mathbf{r}')=\left\{\begin{array}{cc}
                                   f(\mathbf{r})P(\mathbf{r}') & \text{if}\quad \mathbf{r}>\mathbf{r}' \\
                                   f(\mathbf{r}')P(\mathbf{r}) & \text{if} \quad \mathbf{r}'>\mathbf{r}
                                 \end{array}
\right.
\end{equation}
where $P(\mathbf{r})\;(f(\mathbf{r}))$ is the solution  regular (irregular) at the origin of
\begin{equation}\label{eq11}
 \left(T(\mathbf{r})+V_{11}(\mathbf{r})-\frac{\hbar^2k_1^2}{2 \mu}\right)\psi(\mathbf{r})=0, \quad \psi(\mathbf{r})=P(\mathbf{r}),f(\mathbf{r}).
\end{equation}
We can then substitute in (\ref{eq12}) to find

\begin{align}\label{eq13}
\nonumber \left(T(\mathbf{r})+V_{00}(\mathbf{r})-\frac{\hbar^2k_0^2}{2 \mu}\right)\chi_0(\mathbf{r})
 \\+V_{01}(\mathbf{r})\int d\mathbf{r}' f(\mathbf{r}_>)&P(\mathbf{r}_<)V_{10}(\mathbf{r}')\chi_0(\mathbf{r}')=0
\end{align}
\section{Elastic scattering coupled to the 1--n transfer}
Let us now specifically consider the reaction $A(p,p)A$ in which a proton is elastically scattered by a nucleus $A$. In the present context, we will take into account explicitly the coupling with the 1--neutron transfer channel $d+B(=A-1)$. The system will be described with the model wavefunction
\begin{align}\label{eq15}
\Psi(\xi_B,\mathbf  r_n,\mathbf r_p)=\phi_A(\xi_B,\mathbf r_n)\chi_p(\mathbf r_p)+\phi_B(\xi_B)\phi_d(\mathbf r_{pn})\chi_d(\mathbf r_d),
\end{align} 
where the intrinsic wavefunctions are eigenstates of the internal Hamiltonians of the deuteron and the nuclei $A$ and $B$,
\begin{align}\label{eq18}
\nonumber h_d\phi_d&=\epsilon_d\phi_d,\\
\nonumber h_A\phi_A&=\epsilon_A\phi_A,\\
h_B\phi_B&=\epsilon_B\phi_B.
\end{align} 
The Hamiltonian can be written in the $p$-- or in the $d$--channel,
\begin{align}\label{eq14}
\nonumber H&=T_p+h_A+U_p+V_p,\\
H&=T_d+h_d+h_B+U_d+V_d,
\end{align}
with
\begin{align}\label{eq16}
\nonumber V_p&=V_{pn}(r_{pn})+V_{pB}(r_p)-U_p(r_p),\\
U_p&=\braket{\phi_A|(V_{pn}+V_{pB})|\phi_A},
\end{align}
and
\begin{align}\label{eq17}
\nonumber V_d&=V_{pB}(r_p)+V_{nB}(r_n)-U_d(r_d),\\
U_d&=\braket{\phi_B\phi_d|(V_{pB}+V_{nB})|\phi_B\phi_d}.
\end{align}
We will neglect excitations of the core $B$ in the reaction process as well as in the structure of $A$. For simplicity, we will write the wavefunction of the ground state of nucleus $A$  as
\begin{align}
\nonumber \phi_A(\xi_B,\mathbf r_n)=\phi_B(\xi_B)\phi_n(\mathbf r_n)\,,
\end{align}
 If we neglect the coupling to other channels, we have
\begin{align}\label{eq19}
\left(H-E\right)\ket{\Psi}=0.
\end{align}
We note  that
\begin{align}\label{eq157}
\nonumber &\braket{\phi_i|V_i|\phi_i}=0,\\
&\langle\phi_j|(H-E)|\phi_i\rangle=\left\{\begin{array}{l}
\braket{\phi_j|V_i|\phi_i}+\braket{\phi_j|\phi_i}(H_i-E_i) \\ 
\braket{\phi_j|V_j|\phi_i}+(H_j-E_j) \braket{\phi_j|\phi_i},
\end{array}\right.
\end{align}
where we have defined the  channel kinetic energies 
\begin{equation}\label{eq159}
E_i=E-\epsilon_i.
\end{equation}
We can then choose to write the set of coupled equations as,
\begin{align}\label{eq20}
\nonumber &(H_p-E_p)\chi_p=-\bra{\phi_n}V_p\ket{\phi_d}\chi_d-(H_p-E_p)\braket{\phi_n|\phi_d}\chi_d,\\
&(H_d-E_d)\chi_d=-\bra{\phi_d}V_p\ket{\phi_n}\chi_p-\braket{\phi_d|\phi_n}(H_p-E_p)\chi_p,
\end{align}
where $E_p=E-\epsilon_A$, $E_d=E-\epsilon_d-\epsilon_B$ and
with
\begin{align}\label{eq21}
\nonumber H_p&=T_p+U_p,\\
H_d&=T_d+U_d.
\end{align}
If the neutron state $\phi_n$ is bound, we can drop the second term of the right hand side of (\ref{eq20}) (see notes on transfer),
 \begin{align}\label{eq62}
 \nonumber &(H_p-E_p)\chi_p=-\bra{\phi_n}V_p\ket{\phi_d}\chi_d\\
 &(H_d-E_d)\chi_d=-\bra{\phi_d}V_p\ket{\phi_n}\chi_p-\braket{\phi_d|\phi_n}(H_p-E_p)\chi_p,
 \end{align}
or
 \begin{align}\label{eq63}
 \nonumber &(H_p-E_p)\chi_p=-\bra{\phi_n}V_p\ket{\phi_d}\chi_d\\
 &(\widetilde H_d-E_d)\chi_d=-\bra{\phi_d}V_p\ket{\phi_n}\chi_p,
 \end{align}
 with
  \begin{align}\label{eq64}
  \nonumber \widetilde H_d&=H_d-\braket{\phi_d|\phi_n}\bra{\phi_n}V_p\ket{\phi_d}=T_d+U_d-\widetilde U_d,\\
   \widetilde U_d&=\braket{\phi_d|\phi_n}\bra{\phi_n}V_p\ket{\phi_d}.
  \end{align}
  The definition of the non--orthogonality potential $\widetilde U_d$ allows us to cast the problem in the form of the diagonalization of a non--Hermitian matrix,
  \begin{align}\label{eq98}
\left[\begin{matrix}
 H_p-E_p & \bra{\phi_n}V_p\ket{\phi_d} \\ 
 \bra{\phi_d}V_p\ket{\phi_n} & \widetilde H_d-E_d
\end{matrix}\right]\cdot
\left[\begin{matrix}
\chi_p  \\ 
\chi_d
\end{matrix}\right]=0.
  \end{align}  
  It is thus an example of the general problem of the mixing of two unbound levels with widths, as discussed by P. von Brentano, and, more recently, by Laskin et al. in connection to the GPV.
  We can then solve for $\chi_d$,
   \begin{align}\label{eq65}
\chi_d=\widetilde G_d\bra{\phi_d}V_p\ket{\phi_n}\chi_p,
   \end{align} 
   with
      \begin{align}\label{eq66}
\widetilde G_d=\lim_{\epsilon\rightarrow 0}(\widetilde H_d-E_d-i\epsilon)^{-1},
      \end{align} 
the equation for $\chi_p$ can then be written as
      \begin{align}\label{eq67}
(H_p-E_p)\chi_p=-\bra{\phi_n}V_p\ket{\phi_d}\,\widetilde G_d\,\bra{\phi_d}V_p\ket{\phi_n}\chi_p,
      \end{align}
or
\begin{align}\label{eq68}
(\widetilde H_p-E_p)\chi_p=0,
\end{align}      
with
\begin{align}\label{eq69}
\nonumber \widetilde H_p&=T_p+U_p+\widetilde U_p,\\
\widetilde U_p&=\bra{\phi_n}V_p\ket{\phi_d}\,\tilde G_d\,\bra{\phi_d}V_p\ket{\phi_n}.
\end{align}  
We can write down explicitly the non--local potential
  \begin{align}\label{eq52}
\nonumber \widetilde U_p(\mathbf r_p,\mathbf r'_p)&\chi_p(\mathbf r_p)
 =   \int  \phi_n^*(\mathbf r_n)V_p(r_{pn},r_p)\phi_d(\mathbf r_{pn})\\
\times \int \widetilde G_d(\mathbf r_d,\mathbf r_d')&\phi_d^*(\mathbf r'_{pn})V_p(r'_{pn},r'_p)\phi_n(\mathbf r'_n)\chi_p(\mathbf r'_p)d\mathbf r'_{pn}d\mathbf r_d'd\mathbf r_n.
 \end{align}
The non--local term can be made local within the local--momentum approximation,
\begin{align}\label{eq28}
\chi_p(\mathbf r'_p)\approx e^{i\mathbf k_0(r_p)(\mathbf r_p-\mathbf r'_p)}\chi_p(\mathbf r_p),
\end{align}
with 
\begin{align}\label{eq29}
\mathbf k_0(r_p)=\frac{1}{\hbar}\sqrt{2\mu\left(E_p-U_p(r_p)\right)}\,\hat k_p\,.
\end{align}
We can then define a local version of the optical potential,
\begin{align}\label{eq31}
\widetilde U_p(\mathbf r_p,\mathbf r'_p)\chi_p(\mathbf r_p)\approx U^L_p(\mathbf r_p)\chi_p(\mathbf r_p).
\end{align}
Defining 
\begin{align}\label{eq71}
F(\mathbf r_d)=\int \widetilde G_d(\mathbf r_d,\mathbf r_d')&\phi_d^*(\mathbf r'_{pn})V_p(r'_{pn},r'_p)\phi_n(\mathbf r'_n)e^{-i\mathbf k_0(r_p)\mathbf r'_p}d\mathbf r'_{pn}\,  d\mathbf r_d',
\end{align}
we have
 \begin{align}\label{eq32}
 U^L_p(\mathbf r_p) &= e^{i\mathbf k_0(r_p)\mathbf r_p}\int  \phi_n^*(\mathbf r_n)V_p(r_{pn},r_p)\phi_d(\mathbf r_{pn})\,d\mathbf r_nF(\mathbf r_d).
 \end{align}
\subsection{Partial wave decomposition}

The Green's function can be written as
\begin{align}\label{eq44}
\widetilde G_d(\mathbf r_d,\mathbf r_d')=\sum_{l_d}\sqrt{2l_d+1}\frac{f_{l_d}(r_{d<})g_{l_d}(r_{d>})}{k_dr_dr'_d}\left[Y^{l_d}(\hat r_d) Y^{l_d}(\hat r'_d)\right]^0_0,
\end{align}
the plane wave is
\begin{align}\label{eq45}
e^{-i\mathbf k_0(r_p)\mathbf r_p'}=4\pi\sum_{l_p}\sqrt{2l_p+1}\,j_{l_p}(k_0(r_p) r'_p)\left[Y^{l_p}(\hat r'_p) Y^{l_p}(\hat k_p)\right]^0_0,
\end{align}
where $j_{l_p}$ is a spherical Bessel function.  The neutron wavefunctions are
\begin{align}\label{eq38}
\nonumber \phi_n(\mathbf{r}_n)&=u_n(r_n)Y^{l}_{m}(\hat r_n)\\
\phi_d(\mathbf{r}_{pn})&=\frac{1}{\sqrt{4\pi}}u_d(r_{pn}).
\end{align}
The angular part of the integrand in (\ref{eq71}) is
\begin{align}\label{eq46}
\nonumber\left[Y^{l_p}(\hat r'_p) Y^{l_p}(\hat k_p)\right]^0_0&\left[Y^{l_d}(\hat r_d) Y^{l_d}(\hat r'_d)\right]^0_0Y_m^l(\hat r'_n)=\bigl((l_p l_p)_0(l_d l_d)_0|(l_p l_d)_l(l_p l_d)_{l}\bigr)_0\\
\nonumber&\times \left\{\left[Y^{l_p}(\hat r'_p) Y^{l_d}(\hat r'_d)\right]^l\left[Y^{l_d}(\hat r_d) Y^{l_p}(\hat k_p)\right]^l\right\}^0_0Y_m^l(\hat r'_n)\\
\nonumber&=\frac{(-1)^{l-m}}{\sqrt{(2l_p+1)(2l_d+1)}}\left[Y^{l_d}(\hat r_d) Y^{l_p}(\hat k_p)\right]^l_m\left[Y^{l_p}(\hat r'_p) Y^{l_d}(\hat r'_d)\right]^l_{-m}Y_m^l(\hat r'_n)\\
\nonumber &=\frac{1}{\sqrt{(2l+1)(2l_p+1)(2l_d+1)}}\left[Y^{l_d}(\hat r_d) Y^{l_p}(\hat k_p)\right]^l_m\\
&\times\left\{\left[Y^{l_p}(\hat r'_p) Y^{l_d}(\hat r'_d)\right]^lY^l(\hat r'_n)\right\}^0_0,
\end{align}
where we have used
\begin{equation}\label{eq61}
\bigl((l_p l_p)_0(l_d l_d)_0|(l_p l_d)_l(l_p l_d)_{l}\bigr)_0=\sqrt{\frac{2l+1}{(2l_p+1)(2l_d+1)}}.
\end{equation}
We then have 
 \begin{align}\label{eq47}
\nonumber F(\mathbf r_d)=&\frac{\sqrt{4\pi}}{k_dr_d\sqrt{2l+1}}\sum_{l_pl_d} \left[Y^{l_d}(\hat r_d) Y^{l_p}(\hat k_p)\right]^l_m\\
\nonumber&\times\int \left\{\left[Y^{l_p}(\hat r'_p) Y^{l_d}(\hat r'_d)\right]^lY^l(\hat r'_n)\right\}^0_0j_{l_p}(k_0(r_p) r'_p)\\
&\times \frac{f_{l_d}(r_{d<})g_{l_d}(r_{d>})}{r'_d}  u_d(r'_d)u_n(r'_n)V_{pn}(r'_{pn}) d\mathbf r_d' d\mathbf r_{pn}'d\mathbf r_n.
 \end{align}
Note that we have discarded the remnant term and taken $V_{p}\approx V_{pn}$.
Taking $\mathbf r'_d$ and $\mathbf k_p$ in the $z$ direction, we have
\begin{align}\label{eq43}
\nonumber\mathbf r'_d&=r'_d\left(0,0,1\right),\\
\nonumber\mathbf r_{pn}'&=r'_{pn}\left(-\cos\theta,0,\sin\theta\right),\\
\nonumber \mathbf r'_{n}&=\mathbf r'_d-\frac{1}{2}\mathbf r'_{pn}=\left(\frac{1}{2}r'_{pn}\cos\theta,0,r'_d-\frac{1}{2}r'_{pn}\sin\theta\right),\\
 \mathbf r'_{p}&=\mathbf r'_d+\frac{1}{2}\mathbf r'_{pn}=\left(-\frac{1}{2}r'_{pn}\cos\theta,0,r'_d+\frac{1}{2}r'_{pn}\sin\theta\right).
\end{align}
In this specific configuration,
 \begin{align}\label{eq72}
\left[Y^{l_d}(\hat r_d) Y^{l_p}(\hat k_p)\right]^l_m=\sqrt{\frac{2l_p+1}{4\pi}}\langle l_d\,m\,l_p\,0\,|\,l\,m\rangle \,Y^{l_d}_m(\hat r_d),
 \end{align}
 and
  \begin{align}\label{eq73}
  \nonumber \left\{\left[Y^{l_p}(\hat r'_p)Y^{l_d}(\hat r'_d)\right]^lY^l(\hat r'_n)\right\}^0_0&=\sqrt{\frac{2l_d+1}{4\pi}}\frac{(-1)^{l}}{\sqrt{2l+1}}\\
  &\times\sum_{M}(-1)^M\langle l_p\,-M\,l_d\,0\,|\,l\,-M\rangle    Y^{l_p}_{-M}(\hat r'_p)Y^{l}_{M}(\hat r'_n).
  \end{align}
  Taking into account that the integration over the remaining angles (other than $\theta$) yield a factor $8\pi^2$, we have
 \begin{align}\label{eq74}
 \nonumber F(\mathbf r_d)=&\frac{4\pi^{3/2}}{k_dr_d\,(2l+1)}\sum_{l_pl_d}\sqrt{(2l_p+1)(2l_d+1)}\,\langle l_d\,m\,l_p\,0\,|\,l\,m\rangle\,Y^{l_d}_m(\hat r_d)\\
 \nonumber&\times\sum_{M}\langle l_p\,-M\,l_d\,0\,|\,l\,-M\rangle  \int  Y^{l_p}_{-M}(\hat r'_p)Y^{l}_{M}(\hat r'_n)j_{l_p}(k_0(r_p) r'_p)\\
 &\times f_{l_d}(r_{d<})g_{l_d}(r_{d>})u_n(r'_n)u_d(r'_{pn}) V_{pn}(r'_{pn}) r'_dr_{pn}^{'2}\sin\theta\, d r_d'dr'_{pn}d\theta.
 \end{align}
 Rearranging the angular momentum couplings, 
  \begin{align}\label{eq85}
  \nonumber F(\mathbf r_d)=&\frac{4\pi^{3/2}}{k_dr_d\,\sqrt{2l+1}}\sum_{l_pl_d}\sqrt{(2l_p+1)}\,\langle l_d\,m\,l_p\,0\,|\,l\,m\rangle\,Y^{l_d}_m(\hat r_d)\\
  \nonumber&\times  \int  \left[Y^{l_p}(\hat r'_p)Y^{l}(\hat r'_n)\right]^{l_d}_0j_{l_p}(k_0(r_p) r'_p)\\
  &\times f_{l_d}(r_{d<})g_{l_d}(r_{d>})u_n(r'_n)u_d(r'_{pn}) V_{pn}(r'_{pn}) r'_dr_{pn}^{'2}\sin\theta\, d r_d'dr'_{pn}d\theta.
  \end{align}
 If we define
 \begin{align}\label{eq48}
\nonumber I_{l_d,l_p}(r_d)=&\frac{4\pi^{3/2}}{k_dr_d\,\sqrt{2l+1}}\sqrt{(2l_p+1)}\,  \int  \left[Y^{l_p}(\hat r'_p)Y^{l}(\hat r'_n)\right]^{l_d}_0j_{l_p}(k_0(r_p) r'_p)\\
&\times f_{l_d}(r_{d<})g_{l_d}(r_{d>})u_n(r'_n)u_d(r'_{pn}) V_{pn}(r'_{pn}) r'_dr_{pn}^{'2}\sin\theta\, d r_d'dr'_{pn}d\theta.
 \end{align}
 we get
 \begin{align}\label{eq75}
 F(\mathbf r_d)=\sum_{l_d,l_p} \langle l_d\,m\,l_p\,0\,|\,l\,m\rangle\,Y^{l_d}_m(\hat r_d)I_{l_d,l_p}(r_d).
 \end{align} 
 Defining
 \begin{align}\label{eq54}
  B_{l'}^m(r_p)=
 &\frac{1}{\sqrt{4\pi}}\int u_n^*(r_n)V_{pn}(r_{pn})u_d(r_{pn})  Y^{l}_{-m}(\hat r_n)Y^{l'}_0(\hat r_p)F(\mathbf r_d) d\mathbf r_nd\Omega_p,
 \end{align} 
 we have
 \begin{align}\label{eq53}
  U^L_p(\mathbf r_p)=e^{i\mathbf k_0(r_p)\mathbf r_p}\sum_{l'}B^m_{l'}(r_p)Y_0^{l'}(\hat r_p).
 \end{align}
 Taking into account that
  \begin{align}\label{eq76}
 \nonumber Y^{l}_{-m}(\hat r_n)&Y^{l_d}_m(\hat r_d)Y^{l'}_0(\hat r_p)=\langle l\,-m\,l_d\,m\,|\,l'\,0\rangle\frac{(-1)^{l'}}{\sqrt{2l'+1}}\\
 &\times\left\{\left[Y^{l}(\hat r_n)Y^{l_d}(\hat r_d)\right]^{l'}Y^{l'}(\hat r_p)\right\}^0_0,
  \end{align}
we can write
 \begin{align}\label{eq55}
 \nonumber  B_{l'}^m(r_p)=&\frac{(-1)^{l'}}{\sqrt{(2l'+1)4\pi}}\sum_{l_d,l_p}\langle l\,-m\,l_d\,m\,|\,l'\,0\rangle\langle l_d\,m\,l_p\,0\,|\,l\,m\rangle\int u_n^*(r_n)V_{pn}(r_{pn})u_d(r_{pn})\\
 &\times  \left\{\left[Y^{l}(\hat r_n)Y^{l_d}(\hat r_d)\right]^{l'}Y^{l'}(\hat r_p)\right\}^0_0 I_{l_d,l_p}(r_d)d\mathbf r_nd\Omega_p.
 \end{align}
 We can evaluate the above integral in the specific configuration
 \begin{align}\label{eq77}
 \nonumber&\mathbf r_p=r_p\left(0,0,1\right),\\
 \nonumber&\mathbf r_{n}=r_{n}\left(\sin\theta,0,\cos\theta\right),\\
 \nonumber &\mathbf r_{d}=\frac{1}{2}\left(r_{n}\sin\theta,0,r_p+r_{n}\cos\theta\right),\\
&\mathbf r_{pn}=\left(r_{n}\sin\theta,0,r_{n}\cos\theta-r_p\right),
 \end{align}
 in which
 \begin{align}\label{eq78}
  \left\{\left[Y^{l}(\hat r_n)Y^{l_d}(\hat r_d)\right]^{l'}Y^{l'}(\hat r_p)\right\}^0_0=\frac{(-1)^{l'}}{\sqrt{4\pi}}\left[Y^{l}(\hat r_n)Y^{l_d}(\hat r_d)\right]^{l'}_0.
 \end{align}
 Including the $8\pi^2$ factor, we thus have 
  \begin{align}\label{eq56}
  \nonumber B^m_{l'}(r_p)=&\frac{2\pi}{\sqrt{(2l'+1)}}\sum_{l_d,l_p}\langle l\,-m\,l_d\,m\,|\,l'\,0\rangle\langle l_d\,m\,l_p\,0\,|\,l\,m\rangle\int u_n^*(r_n)V_{pn}(r_{pn})u_d(r_{pn})\\
  &\times  \left[Y^{l}(\hat r_n)Y^{l_d}(\hat r_d)\right]^{l'}_0 I_{l_d,l_p}(r_d)r^2_n d r_n \sin\theta d\theta.
  \end{align}
  \subsubsection{Extraction of a central potential}
The local approximation for the polarization correction to the optical potential 
 \begin{equation}\label{eq58}
U^L_p(\mathbf r_p)=e^{i\mathbf k_0(r_p)\mathbf r_p}\sum_{l'} B^m_{l'}(r_p)Y_0^{l'}(\hat r_p),
 \end{equation}
is non--central and $l'$--dependent, but we can evaluate the average over all orientations,
 \begin{equation}\label{eq50}
\overline U^L_p(r_p)=\frac{1}{4\pi}\sum_{l'} B^m_{l'}(r_p)\int e^{i\mathbf k_0(r_p)\mathbf r_p}Y_0^{l'}(\hat r_p)\,d\Omega_p,
 \end{equation}
 and we get
 \begin{equation}\label{eq70}
 \overline U^L_p(r_p)=\sum_{l'}\sqrt{\frac{2l'+1}{4\pi}} B^m_{l'}(r_p)j_{l'}(k_0(r_p) \,r_p),
 \end{equation}
 which, of course, do not depend on the orientation of the proton--target coordinate.
\subsubsection{Average over angular momentum orientations} 
We have an expression for the polarization that result from the coupling to the transfer to a state with a given projection $m$ of the angular momentum $l$. We should average over all possible projections, 
 \begin{equation}\label{eq79}
 \overline U^L_p(r_p)=\frac{1}{2l+1}\sum_{l',m}\sqrt{\frac{2l'+1}{4\pi}} B^m_{l'}(r_p)j_{l'}(k_0(r_p) \,r_p).
 \end{equation}
 If we define 
  \begin{align}\label{eq81}
  \nonumber A_{l_d,l_p}(r_p)=&\frac{2\pi}{\sqrt{(2l'+1)}}\int u_n^*(r_n)V_{pn}(r_{pn})u_d(r_{pn})\\
  &\times  \left[Y^{l}(\hat r_n)Y^{l_d}(\hat r_d)\right]^{l'}_0 I_{l_d,l_p}(r_d)r^2_n d r_n \sin\theta d\theta,
  \end{align}
we can write
\begin{align}\label{eq80}
\nonumber    \sum_m &B^m_{l'}(r_p)=\sum_{l_d,l_p}\,A_{l_d,l_p}(r_p)\sum_m \langle l\,-m\,l_d\,m\,|\,l'\,0\rangle\,\langle l_d\,m\,l_p\,0\,|\,l\,m\rangle \\
&=\sum_{l_d,l_p}\,A_{l_d,l_p}(r_p)\sqrt{\frac{2l+1}{2l_p+1}}\sum_m \langle l\,-m\,l_d\,m\,|\,l'\,0\rangle\,\langle l\,-m\,l_d\,m\,|\,l_p\,0\rangle\\
&=\sum_{l_d,l_p}\,A_{l_d,l_p}(r_p)\sqrt{\frac{2l+1}{2l_p+1}}\,\delta_{l',l_p}.
\end{align}
We can then summarize all the above results with the following expressions and definitions,
 \begin{align}\label{eq83}
\nonumber I_{l_d,l_p}(r_d)=&\frac{4\pi^{3/2}}{k_dr_d\,\sqrt{2l+1}}\sqrt{(2l_p+1)}\,  \int  \left[Y^{l_p}(\hat r'_p)Y^{l}(\hat r'_n)\right]^{l_d}_0j_{l_p}(k_0(r_p) r'_p)\\
&\times f_{l_d}(r_{d<})g_{l_d}(r_{d>})u_n(r'_n)u_d(r'_{pn}) V_{pn}(r'_{pn}) r'_dr_{pn}^{'2}\sin\theta\, d r_d'dr'_{pn}d\theta.     
 \end{align}
 and
 \begin{align}\label{eq82}
 \nonumber B_{l_p}(r_p)=&\frac{\pi^{3/2}}{\sqrt{2l_p+1}}\sum_{l_d}\int u_n^*(r_n)V_{pn}(r_{pn})u_d(r_{pn})\\
 &\times  \left[Y^{l}(\hat r_n)Y^{l_d}(\hat r_d)\right]^{l_p}_0 I_{l_d,l_p}(r_d)r^2_n d r_n \sin\theta d\theta,
 \end{align}
 and finally,
 \begin{equation}\label{eq84}
 \overline U^L_p(r_p)=\frac{1}{\sqrt{2l+1}}\sum_{l_p} B_{l_p}(r_p)j_{l_p}(k_0(r_p) \,r_p),
 \end{equation} 
which is a local and central polarization optical potential.
\subsubsection{The non--orthogonality potential $\widetilde{U}_d$}
The propagator in the deuteron channel contains, aside from the usual optical potential $U_d(r_p)$, a non central non--orthogonality potential $\widetilde{U}_d(\mathbf r_d)$,
  \begin{align}\label{eq86}
  \widetilde U_d(\mathbf r_d)=\braket{\phi_d|\phi_n}\bra{\phi_n}V_p\ket{\phi_d}.
  \end{align}
Let's consider separately the overlap and the potential matrix element,
  \begin{align}\label{eq87}
  \braket{\phi_d|\phi_n}=g_d(r_d)Y^l_m(\hat r_d),
  \end{align}
  and
    \begin{align}\label{eq88}
\bra{\phi_n}V_p\ket{\phi_d}=v_d(r_d)Y^l_{-m}(\hat r_d),
    \end{align}
    with
  \begin{align}\label{eq89}
g_d(r_d)=\frac{1}{\sqrt{4\pi}}\int u_d(r_{pn})u_n(r_n)Y^l_{-m}(\hat r_d)Y^l_{m}(\hat r_n)d\mathbf r_nd\Omega_d,
  \end{align}
and
  \begin{align}\label{eq90}
  v_d(r_d)=\frac{1}{\sqrt{4\pi}}\int u_d(r_{pn})u_n(r_n)V_p(r_{pn})Y^l_{-m}(\hat r_d)Y^l_{m}(\hat r_n)d\mathbf r_nd\Omega_d.
  \end{align}
In order to calculate the overlap $g_d$, we first note that, in order for the integral to be different from zero we must keep only the term coupled to zero total angular momentum, i.e., we can make the replacement
  \begin{align}\label{eq91}
Y^l_{-m}(\hat r_d)Y^l_{m}(\hat r_n)\rightarrow \frac{(-1)^{l+m}}{\sqrt{2l+1}}\left[Y^l(\hat r_d)Y^l(\hat r_n)\right]^0_0.
  \end{align}
The rotationally invariant integrand can then be evaluated in the configuration
\begin{align}\label{eq95}
\nonumber\mathbf r_d&=r_d\left(0,0,1\right),\\
\nonumber\mathbf r_{n}&=r_{n}\left(-\sin\theta,0,\cos\theta\right),\\
 \mathbf r_{pn}&=2(\mathbf r_n-\mathbf r_d)=2\left(-r_{n}\sin\theta,0,r_n\cos\theta-r_{d}\right),
\end{align}
 and the result multiplied by $8\pi^2$. In this configuration,
  \begin{align}\label{eq92}
\frac{(-1)^{l+m}}{\sqrt{2l+1}}\left[Y^l(\hat r_d)Y^l(\hat r_n)\right]^0_0=\frac{1}{\sqrt{4\pi}}\frac{(-1)^m}{\sqrt{2l+1}}Y^l_0(\hat r_n),
  \end{align}
so
  \begin{align}\label{eq93}
  g_d(r_d)=\frac{1}{4\pi}\frac{(-1)^m}{\sqrt{2l+1}}\int u_d(r_{pn})u_n(r_n)Y^l_{0}(\hat r_n)r_n^2\,\sin\theta\,d\theta.
  \end{align}
Similarly,
  \begin{align}\label{eq94}
  v_d(r_d)=\frac{1}{4\pi}\frac{(-1)^m}{\sqrt{2l+1}}\int u_d(r_{pn})u_n(r_n)V_p(r_{pn})Y^l_{0}(\hat r_n)r_n^2\,\sin\theta\,d\theta.
  \end{align}
Then,
  \begin{align}\label{eq96}
 \nonumber  \widetilde U_d(\mathbf r_d)&=g_d(r_d)v_d(r_d) Y^l_m(\hat r_d)Y^l_{-m}(\hat r_d)\\
 &=g_d(r_d)v_d(r_d)\sum_L \langle l\,m\,l\,-m\,|\,L\,0\rangle\left[Y^l(\hat r_d)Y^l(\hat r_d)\right]^L_0.
  \end{align}
This potential is non central, but a central version can be obtained by integrating over the whole $\Omega_d$ angular range and dividing by $4\pi$. We obtain
  \begin{align}\label{eq97}
   \widetilde U_d(r_d)=\frac{1}{4\pi}\,g_d(r_d)v_d(r_d).
  \end{align}







\end{document}















































